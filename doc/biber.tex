\documentclass{ltxdockit}
\usepackage[british]{babel}
\usepackage[strict=true,autostyle=true]{csquotes}
\usepackage{ifthen}
\usepackage{fontspec}
\setmainfont[Ligatures=TeX]{TeXGyrePagella}
\setsansfont{Arial}
\setmonofont{Courier New}

\MakeAutoQuote{«}{»}

\titlepage{%
  title={biber},
  subtitle={A backend bibliography processor for biblatex},
  url={http://biblatex-biber.sourceforge.net},
  author={François Charette, Philip Kime},
  email={firmicus@ankabut.net, Philip@kime.org.uk},
  revision={0.5.5},
  date={\today}}

\hypersetup{%
  pdftitle={biber},
  pdfsubject={A backend bibliography processor for biblatex},
  pdfauthor={Philip Kime},
  pdfkeywords={biblatex, bibliography}}



\def\biberex#1{\hbox{\hspace{-4em}\texttt{\small \detokenize{#1}}}}

\begin{document}

\printtitlepage
\tableofcontents

\section{Introduction}
\label{int}

\subsection{About}

All about \verb=biber=

\subsection{Requirements}\label{ref:req}

Raw perl vs binaries. List binary compatibilities.

\subsection{License}


\subsection{History}

bibtex, Unicode, .bib format etc.

\subsection{Acknowledgments}


\section{Use}\label{ref:use}
\label{use}

.bcf interface etc.

\subsection{Sorting and Localisation}

\verb=biber= takes care of sorting the bibliography for \verb=biblatex=. It writes
entries to the \verb=.bbl= file sorted by a completely customisable set of rules
which are passed in the \verb=.bcf= file by \verb=biblatex=. \verb=biber='s collation code
can run in two different ways\\[2ex]

\biberex{--collate|-C} The default. This option makes \verb=biber= use the
  \verb=Unicode::Collate= module for collation which implements the full UCA (Unicode
  Collation Algorithm). It also has CLDR (Common Locale Data
  Repository) tailoring to deal with cases which are not covered in the
  UCA. See below.\\[1ex]

\biberex{--fastsort|-f} Biber will sort using
  the OS locale collation tables. The drawback for this method is that special
  collation tailoring for various languages are not implemented in the
  collation tables for many OSes. For example, few OSes correctly sort 'å'
  before 'ä' in the Swedish (\verb=sv_SE=) locale. If you are using a more
  common latin alphabet then this is probably not a problem for you.\\[2ex]

\noindent The locale for sorting is determined by the following resource
chain in decreasing precedence order:\\[2ex]

\noindent\verb=--collate_options|-c= $\rightarrow$\\
\hspace*{2em}\verb=--locale|-c= $\rightarrow$\\
\hspace*{4em}\verb=LC_COLLATE= environment variable $\rightarrow$\\
\hspace*{6em}\verb=LANG= environment variable $\rightarrow$\\
\hspace*{8em}\verb=LC_ALL= environment variable\\[2ex]

\noindent The locale will be used with the default \verb=--collate|-C= option
to look for a collation tailoring for that locale. It will generate an
information warning if it finds none. This is not a problem as the vast
majority of collation cases are covered by the basic standard UCA and many
locales neither have nor need any special collation tailoring. The
option \verb=--collate_options|-c= allows users to specify
options to \verb=Unicode::Collate= which tailor the collation in various ways in
addition to the locale tailoring which is automatically performed.
Users should see the the documentation to the module for the various
options, most of which the vast majority of users will never need. The
documentation of the current version is here:\\[2ex]

\noindent\url{http://search.cpan.org/~sadahiro/Unicode-Collate-0.59-withoutworldwriteables/Collate.pm}\\[2ex]

\noindent The most generally useful option is to force sorting of lowercase
before uppercase (with \verb=--collate|-C= \verb=biber= sorts upper case before lower
as this is more common). See the examples below.

\subsubsection{Examples}

\biberex{biber}

\noindent Call biber using all settings from the \verb=.bcf= generated from the
*TeX run. UCA sorting is performed using the \verb=LC_COLLATE= environment
variable for the sorting locale.



\subsection{Limitations}
\label{use:limit}

Custom entry types/fields. List uniqueness etc.


\end{document}
