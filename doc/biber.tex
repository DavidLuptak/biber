\documentclass{ltxdockit}
\usepackage[british]{babel}
\usepackage[strict=true,autostyle=true]{csquotes}
\usepackage{ifthen}
\usepackage{fontspec}
\setmainfont[Ligatures=TeX]{TeXGyrePagella}
\setsansfont{Arial}
\setmonofont{Courier New}

\MakeAutoQuote{«}{»}

\titlepage{%
  title={biber},
  subtitle={A backend bibliography processor for biblatex},
  url={http://biblatex-biber.sourceforge.net},
  author={François Charette, Philip Kime},
  email={firmicus@ankabut.net, Philip@kime.org.uk},
  revision={0.5.5},
  date={\today}}

\hypersetup{%
  pdftitle={biber},
  pdfsubject={A backend bibliography processor for biblatex},
  pdfauthor={Philip Kime},
  pdfkeywords={biblatex, bibliography}}



\def\biberex#1{\hbox{\hspace{-4em}\texttt{\small \detokenize{#1}}}}

\begin{document}

\printtitlepage
\tableofcontents

\section{Introduction}
\label{int}

\subsection{About}

All about \verb+biber+

\subsection{Requirements}\label{ref:req}

Raw perl vs binaries. List binary compatibilities.

\subsection{License}


\subsection{History}

bibtex, Unicode, .bib format etc.

\subsection{Acknowledgments}


\section{Use}\label{ref:use}
\label{use}

.bcf interface etc.

\subsection{Collation and Localisation}

\verb+biber+ takes care of collating (sorting) the bibliography for
\verb+biblatex+. It writes entries to the \verb+.bbl+ file sorted by a
completely customisable set of rules which are passed in the
\verb+.bcf+ file by \verb+biblatex+. \verb+biber+ has two ways of performing
collation\\[2ex]

\biberex{--collate|-C} The default. This option makes \verb+biber+ use the
  \verb+Unicode::Collate+ module for collation which implements the full UCA (Unicode
  Collation Algorithm). It also has CLDR (Common Locale Data
  Repository) tailoring to deal with cases which are not covered by the
  UCA. It is a little slower than \verb+--fastsort|-f+ but the
  advantages are such that it's rarely worth using \verb+--fastsort|-f+\\[1ex]

\biberex{--fastsort|-f} Biber will sort using
  the OS locale collation tables. The drawback for this method is that special
  collation tailoring for various languages are not implemented in the
  collation tables for many OSes. For example, few OSes correctly sort 'å'
  before 'ä' in the Swedish (\verb+sv_SE+) locale. If you are using a
  common latin alphabet, then this is probably not a problem for you.\\[2ex]

\noindent The locale used for sorting is determined by the following resource
chain in decreasing precedence order:\\[2ex]

\noindent\verb+--collate_options|-c+ $\rightarrow$\\
\hspace*{1em}\verb+--locale|-c+ $\rightarrow$\\
\hspace*{2em}\verb+LC_COLLATE+ environment variable $\rightarrow$\\
\hspace*{3em}\verb+LANG+ environment variable $\rightarrow$\\
\hspace*{4em}\verb+LC_ALL+ environment variable\\[2ex]

\noindent With the default \verb+--collate|-C+ option, the locale will
be used to look for a collation tailoring for that locale. It will generate an
information warning if it finds none. This is not a problem as the vast
majority of collation cases are covered by the basic standard UCA and many
locales neither have nor need any special collation tailoring.

\noindent With the \verb+--fastsort|-f+ option, the locale will be
used to locate an OS locale definition to use for the collation. This
may or may not be correctly tailored, depending on the locale and the OS.

\noindent Collation is by default case sensitive. You can turn this
off using the \verb+biber+ option \verb+--sortcase=0+ or set the
\verb+biblatex+ option \verb+sortcase=false+.

\noindent \verb+--collate|-C+ by default collates uppercase before
lower. You can reverse this using the \verb+biber+ option \verb+--sortupper=0+
or set the \verb+biblatex+ option \verb+sortupper=false+.

\noindent There are in fact many options to \verb+Unicode::Collate+
which can tailor the collation in various ways in
addition to the locale tailoring which is automatically performed.
Users should see the the documentation to the module for the various
options, most of which the vast majority of users will never
need\footnote{For details on the various options, see
  \url{http://search.cpan.org/~sadahiro/Unicode-Collate-0.59-withoutworldwriteables/Collate.pm}}.
Options are passed as a single quoted string, each option separated by
comma, each option and value separated by \verb+=>+. See examples.

\subsubsection{Examples}

\biberex{\verb+biber+}

\noindent Call biber using all settings from the \verb+.bcf+ generated from the
LaTeX run. Case sensitive UCA sorting is performed taking the locale
for tailoring from the environment

\biberex{\verb+biber --locale=de_DE+}

\noindent Override the environment locale.

\biberex{\verb+biber --fastsort+}

\noindent Use slightly quicker internal sorting routine. This uses the OS locale
files which may or may not be accurate.

\biberex{\verb+biber --sortcase=0+}

\noindent Case insensitive sorting.

\biberex{\verb+biber --sortupper=0 --collate_options="backwards => 2"+}

\noindent Collate lowercase before upper and collate French accents in
reverse order at UCA level 2.

\subsection{Encoding of \verb+.bib+ and \verb+.bbl+ files}

\verb+biber+ takes care of reencoding the \verb+.bib+ data 

\subsubsection{Examples}


\subsection{Limitations}
\label{use:limit}

Custom entry types/fields. List uniqueness etc.


\end{document}
